\documentclass[10pt,a4paper]{article}




\begin{document}

\title{mlrVST User Guide}
\author{Ewan Hemingway\\
  \texttt{info@ewanhemingway.co.uk}}
\date{\today}
\maketitle


\tableofcontents
\newpage

\section{Installation / System Requirements}

The plugin is developed using the JUCE framework so should work on Windows, Mac and Linux . However the setup and installation may vary from platform to platform. You will need a \textit{relatively} new system with zero-latency drivers to get the most out of it, though older machines may be able to cope too. All shapes and sizes of monome are supported, and can be set up in the app's settings page.

\subsection{Windows}

To install the plugin in Windows, just copy \texttt{mlrVST.dll} to your VST plugins folder. For example, in Ableton Live this folder can be found in \texttt{Options -> Perferences -> File/Folder -> Plugin Sources}.

\subsection{Linux}
TODO

\section{Quickstart}

If you want to just start playing with mlrVST, start by dragging a sample onto the first horizontal strip. This strip maps the sample to one of your monome's rows. The top row is reserved for various functions, but the remaining rows can be used to play back.

So now press a button on the 2nd row to start playing. You should hear that the sample starts playing from the point at which you pressed, e.g. pressing half way along the row starts the sample half way through.

\section{Features}

\section{Advanced: Compiling Yourself}

\end{document}